\documentclass[11pt,addpoints]{exam}
\usepackage{amsfonts,amssymb,amsmath, amsthm}
\usepackage{graphicx}
\usepackage{systeme}
\usepackage{pgf,tikz,pgfplots}
\pgfplotsset{compat=1.15}
\usepgfplotslibrary{fillbetween}
\usepackage{mathrsfs}
\usetikzlibrary{arrows}
\usetikzlibrary{calc}


\pagestyle{headandfoot}

\firstpageheader{Sample Exam (\numpoints\ points)\\ September 26, 2019}{}{Name: \underline{\hspace{2.5in}}}
%\firstpageheadrule

\runningheader{Sample Exam}{}{Page \thepage\ of \numpages}
\runningheadrule

\firstpagefooter{}{}{}
\runningfooter{}{}{}


\begin{document}

\begin{center}
\fbox{\fbox{\parbox{6in}{\centering
No notes, calculators, or other aids are allowed.  Read all directions carefully and write your answers in the space provided.  To receive full credit, you must show all of your work.
}}}
\end{center}




\begin{verbatim}
12 , a0 = 1, a2 = a∗−2 = e

−j π2 , a4 = 2, a−4 = −1, a7 = 5, and ak = 0 for k otherwise.

Find the output Fourier coefficients bk, and specify your answer for all k. Show that b0 = −2,

b2 = −2e^−j π/2 , b−2 = −2e^π/2 , b4 = −4, b−4 = 2, b7 = 0, and bk = 0 otherwise.

ii. (3 pts) Using your results from (A), find the output y(t). Simplify y(t) as much as possible.

Show that y(t) = −2 − 4cos(t− π/2) − 4jsin(2t) − 2e^a(j2t).

iii. (3 pts) If x(t) is delayed by 2, then what are the output coefficients bk now? Show that

b0 = −2, b2 = −2e^−j(π/2 +2), b−2 = −2e^j(π/2 +2), b4 = −4e^−j4, b−4 = 2e^j4, bk = 0 otherwise.






4. Causal.

Consider the following input-output relationships of a system:

(a) (2 pts) y(t) = x(t − 3) − x(3 − t). Is the system causal? Show that the system is not causal.

(b) (4 pts) y(t) = x(t/4). Is the system causal?

(c) (2 pts) y(t) = dx(t)/dt. Is the system causal? Show that the system is causal.

(d) (4 pts) y(t) = cos(2t)x(t − 7) + sin(t)x(t − 1). Is the system causal?

(e) (4 pts) The input-output relationship is given as:

y(t) = x(t + 3a − 2b + 4). Given input-output relationship, determine the values of b in terms of a that will make the system causal.






3. Fourier Transform: LTI Systems Described by LCCDE. (32 pts)

(a) Consider the causal LTI system represented by its input-output relationship:

d^2y(t)/dt2 + 4 dy(t)/dt + 3y(t) = −x(t).

i. (4 pts) Find the frequency response H(jw).

ii. (4 pts) Find the impulse response h(t).

iii. (4 pts) Find the output y(t) when x(t) = e−2tu(t).

(b) A causal LTI system is described by the following differential equation:

dy(t)/dt+ 4y(t) = 9x(t).

i. (4 pts) Find the frequency response H(jω) of this system.

ii. (4 pts) Find the magnitude of the frequency response, |H(jω)|.

iii. (4 pts) Sketch the magnitude of the frequency response (for both positive and negative ω).

iv. (4 pts) Classify this system as low-pass/high-pass/band-pass/band-stop.

v. (4 pts) Find the impulse response h(t) of this system.






2. Complex Numbers - Polar Form and Rectangular Form.

(a) (5 pts) Using the unit circle or formulas for r and θ, convert the following complex numbers in to polar form, z = re^(j*θ). Make sure r > 0 and −π < θ ≤ π:

i. z = (√3)/2 + j(1/2)

ii. z = −2

(b) (5 pts) Using the complex plane or Euler’s formula, convert the following complex numbers in to rectangular form, z = x+ jy:

i. z = 3e^(−j*π)

ii. z = 2e^(j*(π/2))






2. Inverse Laplace Transform. (14 pts)

Find x(t) for given X(s) and ROC. Plot pole-zero plots.

(a) (2 pts) X(s) = 1s2+5s+6 , ROC: Re{s} > −2. Show that x(t) is x(t) = e

−2tu(t)− e−3tu(t).

(b) (4 pts) X(s) = s−3s^2+5s+6 , ROC: −3 < Re{s} < −2.

(c) (4 pts) X(s) = s+2s^2+4s+20 , ROC: Re{s} < −2.

(d) (4 pts) X(s) = ss^2+9 , ROC: Re{s} < 0.






3. LTI System Interconnection and Properties (14 pts)

Consider the same LTI systems in Problem 2. Answer the following questions.

(a) (2 pts) Is system T2 BIBO stable? Using the impulse response test that we discussed in lecture,

show that T2 is BIBO stable with ∫∞ −∞ |h(t)|dt = 2.

(b) (4 pts) Suppose T1 and T3 are connected in parallel. Is the overall system causal? Use the impulse response test that we discussed in lecture.

(c) (4 pts) Suppose T2 and T3 are connected in parallel. Find the overall impulse response h(t) and

then find the overall output y(t) when the input is x(t) = sin(t^2).

(d) (4 pts) Suppose T1 and T3 are connected in series. Is the system causal? Is the system BIBO

stable? Use the impulse response tests.






2. Inverse Laplace Transform. (14 pts)

Find x(t) for given X(s) and ROC. Plot pole-zero plots.

(a) (2 pts) X(s) = 1s2+5s+6 , ROC: Re{s} > −2. Show that x(t) is x(t) = e

−2tu(t)− e−3tu(t).

(b) (4 pts) X(s) = s−3s^2+5s+6 , ROC: −3 < Re{s} < −2.

(c) (4 pts) X(s) = s+2s^2+4s+20 , ROC: Re{s} < −2.

(d) (4 pts) X(s) = ss^2+9 , ROC: Re{s} < 0.






2. Inverse Laplace Transform. (14 pts)

Find x(t) for given X(s) and ROC. Plot pole-zero plots.

(a) (2 pts) X(s) = 1s2+5s+6 , ROC: Re{s} > −2. Show that x(t) is x(t) = e

−2tu(t)− e−3tu(t).

(b) (4 pts) X(s) = s−3s^2+5s+6 , ROC: −3 < Re{s} < −2.

(c) (4 pts) X(s) = s+2s^2+4s+20 , ROC: Re{s} < −2.

(d) (4 pts) X(s) = ss^2+9 , ROC: Re{s} < 0.






3. Fourier Transform: LTI Systems Described by LCCDE. (32 pts)

(a) Consider the causal LTI system represented by its input-output relationship:

d^2y(t)/dt2 + 4 dy(t)/dt + 3y(t) = −x(t).

i. (4 pts) Find the frequency response H(jw).

ii. (4 pts) Find the impulse response h(t).

iii. (4 pts) Find the output y(t) when x(t) = e−2tu(t).

(b) A causal LTI system is described by the following differential equation:

dy(t)/dt+ 4y(t) = 9x(t).

i. (4 pts) Find the frequency response H(jω) of this system.

ii. (4 pts) Find the magnitude of the frequency response, |H(jω)|.

iii. (4 pts) Sketch the magnitude of the frequency response (for both positive and negative ω).

iv. (4 pts) Classify this system as low-pass/high-pass/band-pass/band-stop.

v. (4 pts) Find the impulse response h(t) of this system.






1. Review (16 pts)

(a) Partial Fraction Expansion. (2 pts)

X(s) = −5/(s^2+2s+2). Using the cover-up method, show that X(s) = (−5/2*j)/(s+1+j) + (5/2*j)/(s+1−j).

(b) Partial Fraction Expansion. (4 pts)

X(s) = (s+2)/(s(s+1)2(s+5)). Using the cover-up method.

(c) Magnitude and Phase Equation. (2 pts)

Let X = 1α−jω , where α > 0. Evaluate the magnitude and phase of X. Show that |X| =1√α^2+ω^2and ∠X = tan^−1(ω/α).

(d) Signal Properties. (4 pts)

x(t) = cos(t+ π3). Evaluate P∞. Is x(t) a power signal?

Show that the power of the signal is P∞ = 1/2 , thus making x(t) a power signal.

Hint: If x(t) is periodic with fundamental period To, the power of x(t), P∞ is equivalent to the

average power of x(t) over any interval of length To: P = 1/To ∫ To 0 |x(t)|^2 dt.

(e) Convolution. (4 pts)

Let T denote an LTI system with impulse response p(t + 1) − 2p(t − 1), then find and sketch

y(t) = T [p(−t/2)].

Hint: The pulse signal p(t) = u(t)− u(t− 1).






\end{verbatim}

\end{document}