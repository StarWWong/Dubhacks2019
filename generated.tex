\documentclass[11pt,addpoints]{exam}
\usepackage{amsfonts,amssymb,amsmath, amsthm}
\usepackage{graphicx}
\usepackage{systeme}
\usepackage{pgf,tikz,pgfplots}
\pgfplotsset{compat=1.15}
\usepgfplotslibrary{fillbetween}
\usepackage{mathrsfs}
\usetikzlibrary{arrows}
\usetikzlibrary{calc}


\pagestyle{headandfoot}

\firstpageheader{Sample Exam (\numpoints\ points)\\ September 26, 2019}{}{Name: \underline{\hspace{2.5in}}}
%\firstpageheadrule

\runningheader{Sample Exam}{}{Page \thepage\ of \numpages}
\runningheadrule

\firstpagefooter{}{}{}
\runningfooter{}{}{}


\begin{document}

\begin{center}
\fbox{\fbox{\parbox{6in}{\centering
No notes, calculators, or other aids are allowed.  Read all directions carefully and write your answers in the space provided.  To receive full credit, you must show all of your work.
}}}
\end{center}




\begin{verbatim}
4. Impulse Train Sampling System. (34 pts) The purpose of this problem is to get you comfortable walk-

ing through a block diagram consisting of an impulse train sampling system and to visually understand

the concept of aliasing.

(a) x(t) undergoes impulse train sampling through the following system below:

Answer the following questions:

i. (2 pts) What is the sampling frequency ωs used by this system? What is the equation for the

output Fourier Transform Xs(jω) in terms of X(jω)?

Show that ωs = 10 and Xs(jω) = 5π ∞ ∑ k=−∞ X(j(ω − 10k))

ii. (4 pts) Using your equation from (i), sketch the output spectrum Xs(jω) vs. ω. Make sure

to label all critical points.

iii. (2 pts) Using your sketch from (ii) and your understanding of the concept of aliasing, explain

why this is an example of sampling with no aliasing.

(b) Now consider the following system diagram with a different impulse train and input x(t):

Answer the following questions:

i. (4 pts) What is the sampling frequency fs (not ωs!) used by this system? What is the

equation for the output Fourier Transform Xs(jω) in terms of X(jω)?

ii. (4 pts) Using your equation from (i), sketch the output spectrum Xs(jω) vs. ω. Make sure

to label all critical points.

Note: you need to first find and sketch X(jω).

iii. (2 pts) Using your sketch from (ii), does aliasing of x(t) occur? Justify your answer.

(c) Now consider the following system consisting of both modulation and impulse train sampling

systems:

i. (4 pts) What is the sampling period Ts used by the impulse train sampling system? What

is the equation for X1(jω) in terms of X(jω)? What is the equation for Y (jω) in terms of

X1(jω)?

ii. (4 pts) Sketch the final output spectrum Y (jω) vs. ω. Make sure to label all critical points.

iii. (4 pts) Using your sketch, is it possible to recover the original input x(t) from y(t)? If so,

draw or describe a system diagram that will do this. Make sure to clearly specify all system

blocks being used either in words or with a graph.

(d) (4 pts) Using your knowledge of the behavior of impulse train sampling systems, draw the ap-

propriate system diagram that will produce the following input-output pair. Justify your design

choices.

Make sure to clearly specify your systems with an equation, description, or graph.






4. Fourier Transform: Periodic Signals (8 pts)

(a) (4 pts) Find the Fourier Transform of x(t) = e^jπt + sin(2πt) without using the table of pairs

(applying the formula of Fourier Transform in Continuous Time).

(b) (4 pts) Find the Fourier Transform of x(t) = ∞∑ k=−∞ δ(t− 2k).

Hint: Find the Fourier series coefficients ak of the signal first, and then apply the formula

X(jw) = ∞∑ k=−∞ 2πakδ(w − kwo).






2. Inverse Laplace Transform. (14 pts)

Find x(t) for given X(s) and ROC. Plot pole-zero plots.

(a) (2 pts) X(s) = 1s2+5s+6 , ROC: Re{s} > −2. Show that x(t) is x(t) = e

−2tu(t)− e−3tu(t).

(b) (4 pts) X(s) = s−3s^2+5s+6 , ROC: −3 < Re{s} < −2.

(c) (4 pts) X(s) = s+2s^2+4s+20 , ROC: Re{s} < −2.

(d) (4 pts) X(s) = ss^2+9 , ROC: Re{s} < 0.






0, otherwise

Compute the average power of the signal x(t).

(c) (4 pts) Suppose a signal with Fourier Series representation ωo = 2, a5 = a−5 = − 1/3 , a3 = a−3 =






1. Laplace Transform. (10 pts)

Find the Laplace Transform of the following signals and sketch the corresponding pole-zero plot for

each signal. In the plot, indicate the regions of convergence (ROC). Write X(s) as a single fraction in

the form of N(s)D(s) .

(a) (2 pts) x(t) = e^−4tu(t) + e^−6tu(t). Show that X(s) = 2s+10(s+4)(s+6) . with ROC of Re{s} > −4.

(b) (4 pts) x(t) = e^4tu(−t) + e8tu(−t).

(c) (4 pts) x(t) = δ(t)− u(−t).






1, −2 < w < 0

−1, 0 6 w < 2






4. Fourier Transform: Periodic Signals (8 pts)

(a) (4 pts) Find the Fourier Transform of x(t) = e^jπt + sin(2πt) without using the table of pairs

(applying the formula of Fourier Transform in Continuous Time).

(b) (4 pts) Find the Fourier Transform of x(t) = ∞∑ k=−∞ δ(t− 2k).

Hint: Find the Fourier series coefficients ak of the signal first, and then apply the formula

X(jw) = ∞∑ k=−∞ 2πakδ(w − kwo).






3. Demodulation. (26 pts) The purpose of this problem is to apply what you have learned about mod-

ulation by cosine to design a demodulation system to recover an input or obtain some other desired

result.

(a) Consider the following block diagram:

i. (4 pts) Sketch the output spectrum Y (jω). Precisely label all points and axes.

ii. X(jω) can be recovered from Y (jω) using a single modulator followed by an ideal LPF with

frequency response Hr(jω). Answer the following questions:

A. (4 pts) Draw the block diagram for this system. Clearly specify the modulation signal,

as well as the filter gain A and cutoff frequency wc for the ideal LPF.

B. (2 pts) Note that several cutoff frequencies wc are valid for (A). What range of cutoff

frequencies wc can be used to fully recover x(t) from y(t)?

(b) (4 pts) Suppose the output of a modulation system is given by Y (jω) below. Draw the demodu-

lation system that will recover the desired input again, shown below as Xr(jω):

(c) We briefly learned in class that modulation is useful for different radio stations to send radio

signals out simultaneously. This is a process called frequency division multiplexing, which we will

practice with below.

i. (4 pts) Consider the following frequency division multiplexing system that can be used to

transmit multiple signals at the same time:

Draw the overall output Fourier Transform Y (jω) produced by this system.

ii. (4 pts) Suppose a receiver picks up a signal r(t) whose Fourier Transform is given below:

We can get back the received triangle signal x1(t) using the system below. The idea is to first

filter out the desired signal and then shift it back to the center around frequency ω = 0:

Sketch the output Fourier Transform Xr1(jω) of this system.

iii. (4 pts) You will notice the system has an error in the output amplitude. What change can

be made to the receiver system to get the desired amplitude back up to 1? Note: there are

several answers to this question.






5. (Fourier Series: Analysis) (20 pts) In the following problems, we practice analyzing signals and computing their Fourier Series coefficients:

(a) Consider the continuous-time signal x(t) = 2cos(3t)sin(t)− je^(−j8t) + e^(j6(t−3)).

i. (4 pts) Show the Fourier Series representation of x(t) by finding the fundamental frequency

ω0 and its Fourier Series coefficients ak.

Hint: cos(A)sin(B) = 1/2 [sin(A+B)− sin(A−B)]

ii. (2 pts) Using the result of the previous part, find the power of x(t) using Parseval’s Theorem.

(b) Consider a signal x(t) = sin(3πt) + cos(2πt). Answer the following questions about x(t).

i. (1 pt) What the fundamental frequency ω0 of this signal?

ii. (1 pt)Express x(t) as a sum of complex exponentials using Euler’s formula.

iii. (2 pts) What are the coefficients ak in the Fourier Series representation of x(t)?

iv. (1 pt) The DC component of a signal is defined as its mean value. What is the DC component

of this signal?

(c) Consider the periodic signal described by x(t) = ∞∑ k=−∞ δ(t− 2k)− 14 ∞∑ k=−∞ δ(t+ 2k − 1). 

Answer the following questions about this signal.

i. (1 pt) Find the fundamental frequency, ω0.

ii. (4 pts) Find the DC value c0, and the Fourier series coefficients ck for k 6= 0. Specify the

value of ck for even k and odd k.

(d) (4 pts) Consider a periodic signal of the form 

x(t) = { 1 if 0 ≤ t < 4

−1 4 ≤ t < 8 , with period T = 8.

Compute the Fourier series coefficients of this signal using analysis formulas.






3. Fourier Transform: LTI Systems Described by LCCDE. (32 pts)

(a) Consider the causal LTI system represented by its input-output relationship:

d^2y(t)/dt2 + 4 dy(t)/dt + 3y(t) = −x(t).

i. (4 pts) Find the frequency response H(jw).

ii. (4 pts) Find the impulse response h(t).

iii. (4 pts) Find the output y(t) when x(t) = e−2tu(t).

(b) A causal LTI system is described by the following differential equation:

dy(t)/dt+ 4y(t) = 9x(t).

i. (4 pts) Find the frequency response H(jω) of this system.

ii. (4 pts) Find the magnitude of the frequency response, |H(jω)|.

iii. (4 pts) Sketch the magnitude of the frequency response (for both positive and negative ω).

iv. (4 pts) Classify this system as low-pass/high-pass/band-pass/band-stop.

v. (4 pts) Find the impulse response h(t) of this system.






\end{verbatim}

\end{document}