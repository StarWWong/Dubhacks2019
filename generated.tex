\documentclass[11pt,addpoints]{exam}
\usepackage{amsfonts,amssymb,amsmath, amsthm}
\usepackage{graphicx}
\usepackage{systeme}
\usepackage{pgf,tikz,pgfplots}
\pgfplotsset{compat=1.15}
\usepgfplotslibrary{fillbetween}
\usepackage{mathrsfs}
\usetikzlibrary{arrows}
\usetikzlibrary{calc}


\pagestyle{headandfoot}

\firstpageheader{Sample Exam (\numpoints\ points)\\ September 26, 2019}{}{Name: \underline{\hspace{2.5in}}}
%\firstpageheadrule

\runningheader{Sample Exam}{}{Page \thepage\ of \numpages}
\runningheadrule

\firstpagefooter{}{}{}
\runningfooter{}{}{}


\begin{document}

\begin{center}
\fbox{\fbox{\parbox{6in}{\centering
No notes, calculators, or other aids are allowed.  Read all directions carefully and write your answers in the space provided.  To receive full credit, you must show all of your work.
}}}
\end{center}




\begin{verbatim}
3. ROC and Signal Properties. (10 pts)
(a) (5 pts) Suppose x(t) is a real signal with rational Laplace transform X(s) with the following
properties:
i. X(s) has two poles and one zero, with one pole at s = −1− 2j,
ii. the Fourier transform of e2tx(t) does not exist,
iii.
∫∞
−∞ x(t)dt = −2,
iv.
∫∞
−∞ e
−tx(t)dt = 0.
1
Find x(t) (the time-domain signal).
(b) (5 pts) Signal x(t) has the following properties:
i. X(s) is rational with 1 zero and 2 poles
ii. x(t) is real
iii. X(s) has its zero at s = 1 and a known pole at s = 1 - 2j
iv. Area under x(t) is equal to 1
v. e−3tx(t) is absolutely integrable (so Fourier Transform of e−3tx(t) does exist)
Deduce the expression for signal x(t).





3. Fourier Transform: LTI Systems Described by LCCDE. (32 pts)
1
(a) Consider the causal LTI system represented by its input-output relationship:
d2y(t)
dt2 + 4
dy(t)
dt + 3y(t) = −x(t).
i. (4 pts) Find the frequency response H(jw).
ii. (4 pts) Find the impulse response h(t).
iii. (4 pts) Find the output y(t) when x(t) = e−2tu(t).
(b) A causal LTI system is described by the following differential equation:
dy(t)
dt
+ 4y(t) = 9x(t).
i. (4 pts) Find the frequency response H(jω) of this system.
ii. (4 pts) Find the magnitude of the frequency response, |H(jω)|.
iii. (4 pts) Sketch the magnitude of the frequency response (for both positive and negative ω).
iv. (4 pts) Classify this system as low-pass/high-pass/band-pass/band-stop.
v. (4 pts) Find the impulse response h(t) of this system.





1. Review. (20 pts)
(a) System Properties. To get you ready for Ch.9 in which we will introduce another way to test
system properties, let us review
i. (4 pts) Suppose a system is described by the following I/O relationship:
y(t) =
∞∫
t
x(τ − 1)dτ
Without transforming this system into another system description (e.g. h(t) or H(jω)), is
this system causal? Is this system stable? Justify your answers.
ii. (4 pts) Suppose an LTI system is described by the following impulse response:
h(t) = (δ(t) + 2δ(t− 1)) ∗ (δ(t) − δ(t− 2))
Without transforming this system into another system description (e.g. I/O or H(jω)), is
this system causal? Is this system stable? Justify your answers.
(b) LTI System Output. To get you ready for Ch.9, in which we will introduce another way to find the
output of an LTI system, let us review the three different ways we have learned so far to evaluate
the output.
i. (4 pts) x(t) = e−tu(t) and h(t) = u(t) − 2u(t− 1) + u(t− 3). Using convolution in the time
domain, find y(t). Show that y(t) =






2. Fourier Transform: Analysis (27 pts)
(a) (3 pts) x(t) = 2 + e−|t|.
Show that X(jω) = 4πδ(ω) + 21+ω2
(b) (5 pts) x(t) = rect(2(t+ 34 )) − rect(2(t+
1
4 )).
Show that Xjω = jej
ω
2 sin(ω4 )sinc(
ω
4 ).
(c) (3 pts) x(t) = 12e
−j π4 δ(t− 3) + 12e
j π4 δ(t+ 3).
Show that X(jω) = cos(π4 + 3ω).
(d) (3 pts) x(t) = jπ sin t+
1
π cos(3t).
Show that X(jω) = δ(ω − 1) − δ(ω + 1) + δ(ω − 3) + δ(ω + 3).
(e) (3 pts) x(t) is given as
x(t) = (e−t − e−2t)u(t).
Find X(jω). Show that X(jω) = 1(1+jω)(2+jω) .
(f) (5 pts) x(t) is given as
x(t) = e−3|t| sin t.
Find X(jω). Show that X(jω) = − 3j9+(ω−1)2 +
3j
9+(ω+1)2 .
Hint: X(jω) = 12π (X1(jω) ∗X2(jω))
(g) (5 pts) x(t) is given as
x(t) = 4πsinc(4πt) cos(4πt).
Find X(jω).





0, t < 1
2[t− 1], 1 < t < 3





2. Complex Numbers - Polar Form and Rectangular Form.
(a) (5 pts) Using the unit circle or formulas for r and θ, convert the following complex numbers in to polar
form, z = rejθ. Make sure r > 0 and −π < θ ≤ π:
i. z =
√
3
2 + j
1
2 .
ii. z = −2
(b) (5 pts) Using the complex plane or Euler’s formula, convert the following complex numbers in to rectan-
gular form, z = x+ jy:
i. z = 3e−jπ
ii. z = 2ej
π
2





2, −1 < t < 1





6. Invertible.
Given the following input-output relationships of different systems find out if the system is invertible
or not; if so, derive the inverse. Justify your answer in both cases.
(a) (2 pts) y(t) = x(t/3). Is this system invertible?
Show that it is.
(b) (2 pts) y(t) =
{
x3(t) if t > 2
0 otherwise
. Is this system invertible?
Show that it is NOT invertible.
(c) (4 pts) y(t) = tx(t). Is this system invertible?
(d) (4 pts) y(t) =
(
x(t)
)k
, where k is an integer. Is the system invertible?
Hint : Think of two cases, one when k is odd, second when k is even. It might help to draw a
picture of
(
x(t)
)k
for a couple of small values of k to guess what the answer might be; once you
guess it, you can prove invertibility or non-invertibility for each of the two cases.
(e) (2 pts) y(t) =
{
x(t− 7) if t > 1
8x(t) t ≤ 1 . Is this system invertible or not?
Show that it is invertible.





7. Time-Invariance.
Given input-output relationship of a system, prove whether system is time-invariant.
(a) (2 pts) Consider the system y(t) = x(5t) + sin(x(t)). Is this system time-invariant?
(b) Consider a system T with input x(t) and output y(t) related by:
y(t) = x(t){g(t) + g(t− 1)}
3
i. (2 pts) If g(t) = 1 for all t, show that T is time-invariant using the time-invariance test from
lecture.
ii. (2 pts) If g(t) = t, show that T is not time-invariant by using the time-invariance test from
lecture.
(c) (2 pts) Consider the system T where y(t) = T (x(t)) = x(sin(t)). Is this time-invariant?





7. Fourier Series: Parseval’s Relation and LTI (30 pts)
(a) (2 pts) Suppose a signal x(t) has Fourier Series representation ωo = 2, a5 = a−5 = − 13 , a3 = a−3 =
1
3 , a2 = a
∗
−2 =
j
π . Compute the average power of the signal x(t).
Show that P = 49 +
2
π2
(b) (4 pts) Suppose a signal x(t) has Fourier Series representation ωo = 2 and ak =
{
1
2jk, |k| < 3





\end{verbatim}

\end{document}